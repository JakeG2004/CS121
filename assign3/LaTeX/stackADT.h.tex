\lstdefinestyle{cppstyle}{
  language=C++,
  basicstyle=\ttfamily\small\color{black}, % Set text color to black
  keywordstyle=\color{black}\bfseries,
  commentstyle=\color{black},
  stringstyle=\color{black},
  showstringspaces=false,
  breaklines=false,  % Turn off line wrapping
  frame=lines,
  numbers=left,
  numberstyle=\tiny\color{gray},
  numbersep=5pt,
  tabsize=2,
  backgroundcolor=\color{white}, % Set background color to white
}

\begin{lstlisting}[style=cppstyle, caption={stackADT.h}]
/*
stack.h
A header file for interfacing with a linked list
Jake Gendreau
Feb 26, 2024
*/

#ifndef STACK_H
#define STACK_H

using namespace std;

#include <iostream>

class Stack{
    private:
        struct node{
            string data;
            node* next;
        };
        typedef node* nodePtr;

        nodePtr head;

        int count;

    public:
        //constructor
        Stack(){
            //init
            head = NULL;
            count = 0;
        }

        //deconstructor
        ~Stack(){
            nodePtr p = head;
            nodePtr n;

            while(p != NULL){
                n = p;
                p = p -> next;
                delete n;
            }
        }

        //add node onto the front of the list
        void pushFront(string x);

        //add node onto the front of the list
        void pushBack(string x);

        string pop();

        //delete the first node found with the value x if one exists
        void deleteNode(string x);

        //return the first node found in the list
        string peek();

        //output the values in the nodes, one char per line
        void print();

        //return true if there is a node with the value x
        bool isInList(string x);

        //return count of the number of nodes in the list
        int size();

        //delete list
        void deleteStack();
};

#endif

\end{lstlisting}
